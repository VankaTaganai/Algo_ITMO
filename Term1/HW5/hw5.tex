\documentclass{article}

\usepackage[utf8]{inputenc}
\usepackage[russian]{babel}
\usepackage{amsfonts}
\usepackage{fancyvrb}
\title{Homework}
\date{25-10-2019}
\author{Panov Ivan, M3139}


\begin{document}
	\pagenumbering{gobble}
	
	\maketitle
	\newpage
	\pagenumbering{arabic}
	
	\section*{Задача 1}
	Требуется доказать, что в любой сортирующей сети есть компаратор между $i$ и $i + 1$ позицией. Чтобы доказать данный факт, покажем, что для сети без компаратора между $i$ и $i + 1$ позицей, существует перестановка, на которой сеть не работает. Такой перестановкой может являться отсортированная перестановка из $n$ элементов, но элементы $i$ и $i + 1$ поменяны местами. Разделим мысленно элементы на две части: до $i$ включительно, после $i$. Рассмотрим возможные компараторы: только в первой части(данные компараторы ничего не меняют, так как первая часть отсортирована), только во второй части(данные компараторы ничего не меняют, так как вторая часть отсортирована), компараторы из первой части во вторую, без компоратора между $i$ и $i + 1$(данные компараторы ничего не меняют, так как последовательность отсортирована, кроме элементов $i$ и $i + 1$). Получается, что такая сеть не сортирует все возможные перестановки.
	
	\section*{Задача 2}
	Заметим, что когда мы хотим слить $1$ элемент с $n - 1$, то $1$ элемент может стоять на всех $n$ позициях. Изначально этот единственный элемент стоит на какой-то одной позиции, абозначи его за $a$. Когда мы проведем какие-то компараторы на первом уровне, то $a$ может находиться на какой-то одной из двух позиций. После добавления еще одного уровня компараторов $a$ уже может находиться в одной из четырех позиций. При добалении уровня, количество возможных позиций увеличивается в два раза, следавотельно чтобы покрыть все позиции понабится минимум $\log(n)$ словев компараторов.
	
	\section*{Задача 3}
	Давайте проведем компараторы следующим образом: соединим элементы на позициях $1$ и $2n$, $2$ и $2n - 1$, $3$ и $2n - 2$, и так далее. Теперь рассмотрим, как они будут работать. Будем нумеровать компараторы в порядке их появления, если обходить входные позиции в порядке возрастания индексов. Допустим первые $k - 1$ компараторов не сработали, а $k$-й сработал (совершил обмен), это значит, что все следующие компараторы тоже сработают, так как $a_k > a_{2n-k + 1}$, из отсортированности половин следут для всеx $i \ge k$: $a_i > a_{2n-i+1}$. Тогда в первой половине окажется какой-то префикс первой и префикс второй половины. По построению сначала будут лежать $n$ минимальных элементов. Мы потратили всего один уровень компараторов.
	
	
\end{document}
